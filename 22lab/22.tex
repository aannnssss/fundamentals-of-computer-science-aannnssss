\documentclass[12pt, letterpaper]{article}
\usepackage[T1,T2A]{fontenc}
\usepackage[russian]{babel}
\usepackage[utf8]{inputenc}
\usepackage{amsmath}
\DeclareMathOperator\erf{erf}
\usepackage{listings}
\usepackage{xcolor, graphicx}
\usepackage{float}
\usepackage{gnuplottex}
\usepackage{tikz}
\usepackage{hyperref}
\hypersetup{
    colorlinks=true,
    linkcolor=cyan,
    filecolor=magenta,      
    urlcolor=blue,
    pdfpagemode=FullScreen,
    }

\title{Отчёт по лабораторной работе №22 по курсу “Языки и методы программирования”}
\author{Анна Старостина}
\begin{document}
\maketitle
\begin{description}
\item\textbf{Студент группы:} \underline{М80-108Б-22 Старостина Анна Андреевна, № по списку 18}    
\item\textbf{Контакты e-mail:} \underline{st4ro5tinaa@yandex.ru}
\item\textbf{Работа выполнена:} \underline{«14» апреля 2023 г.}
\item\textbf{Входной контроль знаний с оценкой:} 
\item\textbf{Преподаватель:} \underline{асп. каф. 806 Сахарин Никита Александрович}
\item\textbf{Отчет сдан} \underline{«15» апреля 2023 г.}, \textbf{итоговая оценка:} 
\item\textbf{Подпись преподавателя:} \underline{\hspace{3cm}}
\end{description}
\section{Тема}
Издательская система \TeX{}.
\section{Цель работы}
Получить навыки оформления документов в издательской системе \LaTeX{}.
\section{Задание}
Оформить отчёт об изучении \LaTeX{} на \LaTeX{}.
\section{Оборудование}
\begin{description}
\item\textbf{Ноутбук:} Macbook Pro 2022
\item\textbf{ОП:} 8gb
\item\textbf{SSD:} 512 Gb SSD
\item\textbf{Монитор:} 2560x1600
\end{description}
\section{Программное обеспечение}
\begin{description}
\item\textbf{Операционная система семейства:} Ubuntu UTM 19.36.34
\item\textbf{Интерпретатор команд:} bash
\end{description}
\section{Идея, метод, алгоритм решения задачи}
Прочитать документацию \LaTeX{} и переписать отчет с Markdown на \LaTeX{}. tex-файл скомпилировать с помощью утилиты pdflatex. Продемонстрировать пример формул, графиков, фигур. 
\section{Сценарий выполнения работы}
Ознакомиться с литературой по \LaTeX{}. Изучение примеров, находящихся в открытом доступе. Верстка ответа через Online Latex Editor Overleaf.
Продемонстрируем широкий функционал \LaTeX{} на следующих примерах.
\section{Распечатка протокола}
\subsection{Пример формул}
\[\sin (2 \alpha) = 2*\sin \alpha*\cos \alpha\]
\[\int_0^\infty e^{-x}\,\mathrm{d}x\]
\[\displaystyle\sum_{i=1}^{10} t_i\]
\[A=
\begin{pmatrix}
19 & 28 & 37\\
6 & 5 & 4
\end{pmatrix}\]
\[\frac{\frac{1}{x}+\frac{1}{y}}{y-z}\]
\subsection{Пример графика} 
%\documentclass[a4paper]{article}
%\usepackage{gnuplottex}          % <- Use if running MiKTeX.
%\usepackage[miktex]{gnuplottex} % <- Use instead if running TeX Live.

\begin{gnuplot}[terminal=pdf,terminaloptions={font ",10" linewidth 3}]
    plot sin(x), cos(x)
\end{gnuplot}

\begin{gnuplot}[scale=0.8]
    set grid
    set title 'gnuplottex test $e^x$'
    set ylabel '$y$'
    set xlabel '$x$'
    plot exp(x) with linespoints
\end{gnuplot}

\subsection{Пример фигуры} 
\begin{tikzpicture}[xscale=2.0, yscale=0.8]
\draw[dashed, ultra thick] (0,0) -- (3,0) -- (3,4) -- cycle;
\end{tikzpicture}
\section{Дневник отладки}
\begin{tabular}{|c|p{1cm}|p{2cm}|c|p{3 cm}|p{1.5 cm}|p{2.3 cm}|}
    \hline
    № & Лаб. или дом. & Дата & Время & Событие & Действие по исправлению & Примечание\\
    \hline
    1 & Дом. & 14.04.2022 & 21:00 & Выполнение лабораторной работы & - & -\\
    \hline
\end{tabular}
\section{Замечания автора по существу работы}

\section{Выводы}
Были изучены основы оформления докладов в издательской системе \LaTeX{}. Это показалось полезным и интересным, планирую использовать это в дальнейших работах.  \\
\flushright \textbf{Подпись студента:} \underline{\hspace{3cm}}
\end{document}
